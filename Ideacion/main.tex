\documentclass{article}
\usepackage[utf8]{inputenc}
\usepackage[spanish]{babel}
\usepackage{listings}
\usepackage{graphicx}
\graphicspath{ {images/} }
\usepackage{cite}

\begin{document}

\begin{titlepage}
    \begin{center}
        \vspace*{1cm}
            
        \Huge
        \textbf{Proyecto Final}
            
        \vspace{0.5cm}
        \LARGE
        Ideación y planteamiento del proyecto
            
        \vspace{1.5cm}
            
        \textbf{Nelson Fernando Parra Guardia}
            
        \vfill
            
        \vspace{0.8cm}
            
        \Large
        Despartamento de Ingeniería Electrónica y Telecomunicaciones\\
        Universidad de Antioquia\\
        Medellín\\
        Marzo de 2021
            
    \end{center}
\end{titlepage}

\newpage
\section{Descripción del Proyecto}\label{description}
El proyecto será un videojuego (por ahora llamado StonerIt) que consistirá en ir pasando ciertos niveles para lograr completar el mismo. La modalidad consistirá en un personaje que irá derrotando enemigos limitados en algunos minimapas. Al derrotar dichos enemigos, el personaje irá completando un conjunto de minimapas; donde, en alguno de ellos, se encontrará con un jefe. Al terminar la serie de minimapas, el personaje completará el nivel y pasará al siguiente con más dificultad.
Durante la pertida, el personaje podrá tomar poderes del suelo, así como también se proporcionan opciones para aumentar la vida del personaje.

\section{Información Adicional}\label{adinfo}
StonerIt está inspirado (no recreado) en otro videojuego ya existente para dispositivos moviles llamado Archero \cite{habby}. StonerIt variará en cuanto a los poderes otorgados al personaje, así como también en la administración y propósito del puntaje.

\bibliographystyle{IEEEtran}
\bibliography{references}

\end{document}
